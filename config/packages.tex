% ==============================================================================
% PAQUETES NECESARIOS
% ==============================================================================

% ------------------------------------------------------------------------------
% Codificación y fuentes
% ------------------------------------------------------------------------------
\usepackage[utf8]{inputenc}         % Codificación de entrada UTF-8
\usepackage[T1]{fontenc}            % Codificación de fuentes
\usepackage[spanish,es-tabla]{babel} % Idioma español

% ------------------------------------------------------------------------------
% Matemáticas
% ------------------------------------------------------------------------------
\usepackage{amsmath}                % Entornos matemáticos avanzados
\usepackage{amssymb}                % Símbolos matemáticos adicionales
\usepackage{amsthm}                 % Teoremas y demostraciones
\usepackage{mathtools}              % Herramientas matemáticas extendidas

% ------------------------------------------------------------------------------
% Gráficos y figuras
% ------------------------------------------------------------------------------
\usepackage{graphicx}               % Inclusión de imágenes
\usepackage{float}                  % Control de posicionamiento de flotantes
\usepackage{subcaption}             % Subfiguras y subcaptiones
\usepackage{tikz}                   % Creación de gráficos vectoriales
\usepackage{pgfplots}               % Gráficos de funciones
\pgfplotsset{compat=1.18}

% ------------------------------------------------------------------------------
% Tablas
% ------------------------------------------------------------------------------
\usepackage{booktabs}               % Tablas profesionales
\usepackage{multirow}               % Celdas de múltiples filas
\usepackage{multicol}               % Columnas múltiples
\usepackage{longtable}              % Tablas que ocupan varias páginas
\usepackage{array}                  % Mejoras para arrays y tablas

% ------------------------------------------------------------------------------
% Colores
% ------------------------------------------------------------------------------
\usepackage{xcolor}                 % Manejo avanzado de colores
\definecolor{azulmetro}{RGB}{0,51,102}
\definecolor{verdeenergia}{RGB}{46,125,50}
\definecolor{grisoscuro}{RGB}{66,66,66}

% ------------------------------------------------------------------------------
% Hipervínculos y referencias
% ------------------------------------------------------------------------------
\usepackage{hyperref}               % Hipervínculos en el documento
\hypersetup{
    colorlinks=true,
    linkcolor=azulmetro,
    filecolor=verdeenergia,
    urlcolor=azulmetro,
    citecolor=azulmetro,
    pdfauthor={Autor del Proyecto},
    pdftitle={Generación de Energía Piezoeléctrica en Metro},
    pdfsubject={Formulación de Proyecto},
    pdfkeywords={piezoelectricidad, metro, energía renovable, transporte público}
}

% ------------------------------------------------------------------------------
% Bibliografía y citas
% ------------------------------------------------------------------------------
\usepackage{cite}                   % Mejoras en las citas

% ------------------------------------------------------------------------------
% Formato de página y espaciado
% ------------------------------------------------------------------------------
\usepackage{geometry}               % Configuración de márgenes
\geometry{
    a4paper,
    left=3cm,
    right=2.5cm,
    top=2.5cm,
    bottom=2.5cm
}
\usepackage{setspace}               % Control del interlineado
\onehalfspacing                     % Interlineado de 1.5

% ------------------------------------------------------------------------------
% Encabezados y pies de página
% ------------------------------------------------------------------------------
\usepackage{fancyhdr}               % Personalización de encabezados
\pagestyle{fancy}
\fancyhf{}
\fancyhead[LE,RO]{\thepage}
\fancyhead[RE]{\leftmark}
\fancyhead[LO]{\rightmark}
\renewcommand{\headrulewidth}{0.4pt}

% ------------------------------------------------------------------------------
% Listas y enumeraciones
% ------------------------------------------------------------------------------
\usepackage{enumitem}               % Personalización de listas
\setlist[itemize]{leftmargin=*}
\setlist[enumerate]{leftmargin=*}

% ------------------------------------------------------------------------------
% Código fuente
% ------------------------------------------------------------------------------
\usepackage{listings}               % Inclusión de código fuente
\lstset{
    basicstyle=\ttfamily\small,
    breaklines=true,
    frame=single,
    numbers=left,
    numberstyle=\tiny,
    backgroundcolor=\color{gray!10}
}

% ------------------------------------------------------------------------------
% Unidades y símbolos
% ------------------------------------------------------------------------------
\usepackage{siunitx}                % Formato de unidades del SI
\sisetup{
    output-decimal-marker={,},
    per-mode=symbol
}

% ------------------------------------------------------------------------------
% Otros paquetes útiles
% ------------------------------------------------------------------------------
\usepackage{csquotes}               % Citas textuales
\usepackage{lipsum}                 % Texto de relleno (eliminar en versión final)
\usepackage{comment}                % Comentarios de múltiples líneas
