% Fundamentos de la piezoelectricidad

\subsection{Definición del Efecto Piezoeléctrico}

\begin{definicion}
La \piezo{} es la capacidad de ciertos materiales cristalinos de generar una diferencia de potencial eléctrico (voltaje) cuando son sometidos a esfuerzos mecánicos (presión, compresión, torsión o flexión). Este fenómeno también opera de manera inversa: al aplicar un campo eléctrico, el material experimenta deformación mecánica.
\end{definicion}

El término proviene del griego \textit{piezein}, que significa ``presionar'' o ``apretar''.

\subsection{Ecuaciones Fundamentales}

Las ecuaciones constitutivas piezoeléctricas relacionan las variables eléctricas y mecánicas:

\textbf{Efecto directo (mecánico → eléctrico):}
\begin{equation}
D_i = d_{ijk} \sigma_{jk} + \epsilon_{ij}^{\sigma} E_j
\label{eq:piezo_directo}
\end{equation}

\textbf{Efecto inverso (eléctrico → mecánico):}
\begin{equation}
\varepsilon_{ij} = s_{ijkl}^E \sigma_{kl} + d_{kij} E_k
\label{eq:piezo_inverso}
\end{equation}

donde:
\begin{itemize}
    \item $D_i$ es el desplazamiento eléctrico
    \item $d_{ijk}$ es el coeficiente piezoeléctrico
    \item $\sigma_{jk}$ es el tensor de esfuerzos mecánicos
    \item $\epsilon_{ij}^{\sigma}$ es la permitividad dieléctrica
    \item $E_j$ es el campo eléctrico
    \item $\varepsilon_{ij}$ es la deformación mecánica
    \item $s_{ijkl}^E$ es el tensor de compliancia elástica
\end{itemize}

\subsection{Parámetros Clave}

\begin{itemize}
    \item \textbf{Coeficiente piezoeléctrico ($d_{ij}$):} Relaciona la carga generada con la fuerza aplicada. Se mide en picocoulombs por newton (pC/N).
    
    \item \textbf{Constante de voltaje ($g_{ij}$):} Relaciona el campo eléctrico generado con el esfuerzo aplicado. Se mide en volt-metros por newton (Vm/N).
    
    \item \textbf{Factor de acoplamiento electromecánico ($k$):} Mide la eficiencia de conversión entre energía mecánica y eléctrica.
\end{itemize}
