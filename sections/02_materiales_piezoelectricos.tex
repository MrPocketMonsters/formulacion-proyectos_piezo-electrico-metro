% Materiales piezoeléctricos

\subsection{Clasificación de Materiales}

Los materiales piezoeléctricos se clasifican en tres categorías principales:

\subsubsection{Cristales Naturales}

\begin{itemize}
    \item \textbf{Cuarzo (SiO$_2$):} Primer material piezoeléctrico descubierto, estable térmicamente, bajo coeficiente piezoeléctrico.
    \item \textbf{Turmalina:} Fuerte efecto piezoeléctrico, alta resistencia térmica.
    \item \textbf{Sal de Rochelle:} Alto coeficiente piezoeléctrico, baja estabilidad.
\end{itemize}

\subsubsection{Cerámicas Piezoeléctricas}

Las cerámicas son los materiales más utilizados en aplicaciones de energy harvesting debido a su alto rendimiento y bajo costo.

\begin{itemize}
    \item \textbf{Titanato de Zirconato de Plomo (PZT):}
        \begin{itemize}
            \item Material más común en aplicaciones comerciales
            \item Alto coeficiente piezoeléctrico ($d_{33}$ = 300-600 pC/N)
            \item Excelente factor de acoplamiento ($k_{33} \approx 0.70$)
            \item Limitación: contenido de plomo (restricciones ambientales)
        \end{itemize}
    
    \item \textbf{Titanato de Bario (BaTiO$_3$):}
        \begin{itemize}
            \item Alternativa libre de plomo
            \item Coeficiente moderado ($d_{33} \approx 190$ pC/N)
            \item Menor costo de producción
        \end{itemize}
    
    \item \textbf{Niobato de Litio (LiNbO$_3$):}
        \begin{itemize}
            \item Excelente estabilidad térmica
            \item Aplicaciones en sensores de alta temperatura
        \end{itemize}
\end{itemize}

\subsubsection{Polímeros Piezoeléctricos}

\begin{itemize}
    \item \textbf{PVDF (Fluoruro de Polivinilideno):}
        \begin{itemize}
            \item Flexible y ligero
            \item Resistente a impactos
            \item Coeficiente bajo ($d_{33} \approx 20$-30 pC/N)
            \item Ideal para superficies curvas y aplicaciones portátiles
        \end{itemize}
    
    \item \textbf{P(VDF-TrFE):} Copolímero mejorado del PVDF con mejor respuesta piezoeléctrica
\end{itemize}

\subsection{Comparación de Materiales}

\begin{table}[H]
\centering
\begin{tabular}{@{}lcccc@{}}
\toprule
\textbf{Material} & \textbf{$d_{33}$ (pC/N)} & \textbf{$k_{33}$} & \textbf{Costo} & \textbf{Durabilidad} \\ \midrule
PZT & 300-600 & 0.70 & Medio & Alta \\
BaTiO$_3$ & 190 & 0.50 & Bajo & Media \\
PVDF & 20-30 & 0.12 & Alto & Alta \\
Cuarzo & 2.3 & 0.10 & Bajo & Muy Alta \\ \bottomrule
\end{tabular}
\caption{Comparación de propiedades de materiales piezoeléctricos}
\label{tab:materiales_comparacion}
\end{table}

\subsection{Materiales Emergentes}

\begin{itemize}
    \item \textbf{Cerámicas libres de plomo:} KNN (Niobato de Potasio y Sodio), BNT (Titanato de Bismuto y Sodio)
    \item \textbf{Nanocompuestos:} Combinación de nanopartículas cerámicas en matriz polimérica
    \item \textbf{Materiales 2D:} Disulfuro de molibdeno (MoS$_2$), nitruro de boro hexagonal
\end{itemize}

\subsection{Selección de Material para Energy Harvesting}

Para aplicaciones de captación de energía en pisos, los criterios clave son:

\begin{enumerate}
    \item Alta resistencia mecánica y durabilidad
    \item Coeficiente piezoeléctrico elevado
    \item Bajo costo para implementación a gran escala
    \item Cumplimiento de normativas ambientales (preferencia por materiales libres de plomo)
    \item Facilidad de integración en estructuras de piso
\end{enumerate}
