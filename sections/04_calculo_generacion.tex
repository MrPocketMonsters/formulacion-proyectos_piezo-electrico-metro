% Cálculo de generación energética

\subsection{Parámetros de Cálculo}

\subsubsection{Datos de Entrada}

\begin{itemize}
    \item Flujo de pasajeros: 50,000 pasajeros/día promedio
    \item Peso promedio por pasajero: 70 kg
    \item Pasos por pasajero sobre zona piezoeléctrica: 8-10 pasos
    \item Energía por paso (tecnología seleccionada): 2.5 W promedio
    \item Tiempo de generación por paso: 0.5 segundos
    \item Eficiencia del sistema: 75\% (considerando pérdidas)
\end{itemize}

\subsection{Modelo de Cálculo}

\subsubsection{Energía por Pisada}

\begin{equation}
E_{pisada} = P_{pico} \times t_{contacto} \times \eta_{sistema}
\end{equation}

donde:
\begin{itemize}
    \item $E_{pisada}$ = Energía generada por pisada
    \item $P_{pico}$ = Potencia pico durante la pisada (2.5 W)
    \item $t_{contacto}$ = Tiempo de contacto (0.5 s)
    \item $\eta_{sistema}$ = Eficiencia del sistema (0.75)
\end{itemize}

\begin{equation}
E_{pisada} = 2.5 \text{ W} \times 0.5 \text{ s} \times 0.75 = 0.9375 \text{ Wh} = 0.94 \text{ Wh}
\end{equation}

\subsubsection{Generación Diaria}

\begin{equation}
E_{diaria} = N_{pasajeros} \times N_{pasos} \times E_{pisada}
\end{equation}

\begin{equation}
E_{diaria} = 50,000 \times 9 \times 0.94 \text{ Wh} = 423,000 \text{ Wh} = 423 \text{ kWh/día}
\end{equation}

\subsubsection{Generación Anual}

Considerando variación estacional (factor de corrección 0.85):

\begin{equation}
E_{anual} = E_{diaria} \times 365 \times 0.85 = 423 \times 365 \times 0.85 = 131,239 \text{ kWh/año}
\end{equation}

\subsection{Resultados Estimados}

\begin{table}[H]
\centering
\begin{tabular}{@{}lr@{}}
\toprule
\textbf{Concepto} & \textbf{Valor} \\ \midrule
Energía por pisada & 0.94 Wh \\
Generación diaria & 423 kWh \\
Generación mensual & 12,690 kWh \\
Generación anual & 131,239 kWh \\
Potencia pico instalada & 12 kW \\
Factor de capacidad & 15\% \\ \bottomrule
\end{tabular}
\caption{Estimación de generación energética del sistema}
\end{table}

\subsection{Validación del Modelo}

Los resultados estimados son consistentes con estudios similares reportados en la literatura y proyectos implementados internacionalmente.
