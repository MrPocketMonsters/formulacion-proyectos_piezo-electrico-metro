% Planteamiento del problema

El sistema de metro enfrenta diversos desafíos relacionados con el consumo energético y la sostenibilidad:

\begin{enumerate}
    \item \textbf{Alto consumo energético:} Las estaciones del metro requieren grandes cantidades de energía eléctrica para su operación diaria, representando un costo operativo significativo.
    
    \item \textbf{Dependencia de fuentes convencionales:} La energía consumida proviene principalmente de fuentes no renovables, contribuyendo a la huella de carbono del sistema de transporte.
    
    \item \textbf{Desperdicio de energía cinética:} El tránsito de millones de pasajeros genera una cantidad considerable de energía mecánica que actualmente se disipa sin aprovechamiento.
    
    \item \textbf{Limitada implementación de energías renovables:} A pesar de la disponibilidad de diversas tecnologías de energía limpia, su integración en la infraestructura del metro es limitada.
    
    \item \textbf{Presión por sostenibilidad:} Existe una creciente demanda social e institucional por implementar prácticas ambientalmente responsables en el transporte público.
\end{enumerate}

La pregunta central que guía este proyecto es: \textit{¿Es técnica y económicamente viable implementar sistemas de generación de energía piezoeléctrica en las estaciones del metro para reducir el consumo energético convencional y mejorar la sostenibilidad del sistema?}
