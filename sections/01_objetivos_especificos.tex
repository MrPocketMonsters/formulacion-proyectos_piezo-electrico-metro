% Objetivos Específicos

\begin{enumerate}
    \item Realizar un análisis detallado del flujo de pasajeros y las características físicas de las estaciones del metro para identificar las zonas óptimas de instalación de sistemas piezoeléctricos.
    
    \item Evaluar y seleccionar la tecnología piezoeléctrica más adecuada considerando factores de eficiencia, durabilidad, costos y compatibilidad con la infraestructura existente.
    
    \item Calcular el potencial de generación energética de los sistemas piezoeléctricos en función del tráfico peatonal y las características técnicas de los materiales seleccionados.
    
    \item Diseñar el sistema de captación, almacenamiento y distribución de la energía generada, incluyendo su integración con la red eléctrica de las estaciones.
    
    \item Realizar un análisis económico-financiero completo del proyecto, incluyendo costos de inversión, operación, mantenimiento, y proyección de ahorros energéticos.
    
    \item Evaluar el impacto ambiental del proyecto en términos de reducción de emisiones de carbono y contribución a los objetivos de sostenibilidad del sistema de transporte.
    
    \item Desarrollar un plan de implementación detallado que incluya cronograma, recursos necesarios, fases del proyecto y estrategias de mitigación de riesgos.
    
    \item Establecer indicadores de desempeño (KPIs) para el monitoreo y evaluación del sistema una vez implementado.
\end{enumerate}
