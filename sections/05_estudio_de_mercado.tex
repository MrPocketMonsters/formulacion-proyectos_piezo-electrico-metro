% Sección: Estudio de Mercado y Demanda
% --------------------------------------------------
% Este archivo contiene el esqueleto de subsecciones para el estudio de
% mercado. NO INCLUIR contenido final en este punto: sólo comentarios que
% % describen el contenido esperado, entregables y notas metodológicas.

\subsection{Objetivos y alcance del estudio}
% - Comentario: Describir alcance espacial (lista de estaciones), horizonte
%   temporal, unidades de análisis y objetivos concretos del estudio.
%   Entregable: mapa/tabla de estaciones y objetivos cuantificados.

\subsection{Metodología y fuentes}
% - Comentario: Explicar métodos de recolección (conteos manuales/automáticos,
%   encuestas, datos del operador), periodos de muestreo, supuestos y nivel
%   de confianza. Incluir nota sobre tratamiento de datos y errores.

\subsection{Caracterización del mercado y segmentos de usuarios}
% - Comentario: Perfil de usuarios (pendulares, ocasionales, turistas),
%   patrones horarios, comportamiento relevante para instalación y uso.
%   Entregable: gráficos de distribución por horario/segmento.

\subsection{Conteo y análisis de flujos peatonales}
% - Comentario: Indicar tablas/series de conteos (promedio diario, picos
%   horarios, variaciones estacionales). Señalar dónde estarán las áreas
%   instalables y la metodología para convertir pisadas en eventos útiles.

\subsection{Competencia y tecnologías alternativas}
% - Comentario: Comparativa breve entre piezo y otras soluciones (solar,
%   mejora de eficiencia, baterías). Incluir ventajas/desventajas y coste
%   relativo.

\subsection{Factores de mercado y regulatorios}
% - Comentario: Listar normativas, requisitos del operador del metro,
%   incentivos, aceptación pública y restricciones que afectan adopción.

\subsection{Conclusión del estudio de mercado}
% - Comentario: Síntesis final con 4–6 puntos clave y recomendación
%   operacional (seguir a fase de pilotaje / ajustar diseño / descartar).

\subsection{Anexos de datos}
% - Comentario: Indicar aquí referencias a anexos donde estarán las tablas
%   extensas, hojas de cálculo y resultados de conteos crudos. Recomendado
%   mover tablas grandes a `anexos.tex` o `tables/`.

% Notas generales:
% - Mantener el cuerpo del capítulo sintético; dejar los datos extensos en
%   anexos. Cada subsección debe enlazar explícitamente con los supuestos
%   definidos en la sección "Supuestos y restricciones".
% - Priorizar la recolección de conteos en estaciones candidatas; este
%   ítem suele tener el mayor impacto en los resultados.
