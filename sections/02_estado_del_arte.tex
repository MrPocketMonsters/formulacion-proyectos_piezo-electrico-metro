% Estado del arte

\subsection{Implementaciones Internacionales}

\subsubsection{Estación de Shibuya, Tokio (Japón, 2008)}

\begin{itemize}
    \item \textbf{Ubicación:} Tornequet de acceso a la estación
    \item \textbf{Tecnología:} Losetas piezoeléctricas de 45 × 45 cm
    \item \textbf{Generación:} Aproximadamente 0.5-1 W por paso
    \item \textbf{Aplicación:} Alimentación de sistemas de señalización digital
    \item \textbf{Resultado:} Proyecto piloto exitoso, demostró viabilidad técnica
\end{itemize}

\subsubsection{Estación de Rotterdam Central (Países Bajos, 2009-2011)}

\begin{itemize}
    \item \textbf{Empresa:} Sustainable Dance Club (Energy Floors)
    \item \textbf{Área instalada:} 100 m² de piso piezoeléctrico
    \item \textbf{Tráfico:} Aproximadamente 150,000 pasajeros/día
    \item \textbf{Generación estimada:} 16 kWh/mes
    \item \textbf{Aplicación:} Iluminación LED de señalización
    \item \textbf{Innovación:} Sistema interactivo que visualiza la energía generada en tiempo real
\end{itemize}

\subsubsection{Estación Victoria, Londres (Reino Unido, 2009)}

\begin{itemize}
    \item \textbf{Empresa:} Pavegen Systems
    \item \textbf{Tecnología:} Losetas triangulares electromagnéticas (no estrictamente piezoeléctricas)
    \item \textbf{Generación:} 5 W por paso
    \item \textbf{Instalación:} 12 losetas en zona de alto tráfico
    \item \textbf{Resultado:} Energía suficiente para alimentar señalización ambiental
\end{itemize}

\subsubsection{Aeropuerto de Toulouse (Francia, 2013)}

\begin{itemize}
    \item Implementación de tecnología similar en terminal de pasajeros
    \item Generación de hasta 60 vatios durante horas pico
\end{itemize}

\subsection{Investigaciones Académicas Recientes}

\subsubsection{Optimización de Sistemas}

\begin{itemize}
    \item Estudios del MIT (2019): Mejora de eficiencia mediante configuraciones en serie/paralelo de elementos piezoeléctricos
    \item Universidad de Wisconsin (2020): Desarrollo de algoritmos de gestión inteligente de energía (MPPT)
    \item ETH Zurich (2021): Análisis de durabilidad bajo ciclos de carga repetitivos
\end{itemize}

\subsubsection{Nuevos Materiales}

\begin{itemize}
    \item Investigación en materiales flexibles de alto rendimiento
    \item Desarrollo de nanocompuestos con mejor relación eficiencia/costo
    \item Cerámicas piezoeléctricas libres de plomo con propiedades mejoradas
\end{itemize}

\subsection{Tendencias Tecnológicas}

\begin{enumerate}
    \item \textbf{Hibridación:} Combinación de captación piezoeléctrica con otras tecnologías (solar, cinética)
    \item \textbf{IoT Integration:} Integración con Internet de las Cosas para monitoreo en tiempo real
    \item \textbf{Smart Grids:} Conexión inteligente con redes eléctricas para optimizar distribución
    \item \textbf{Materiales sostenibles:} Transición hacia materiales ecológicos y reciclables
    \item \textbf{Modularidad:} Diseños modulares para facilitar instalación y mantenimiento
\end{enumerate}

\subsection{Lecciones Aprendidas}

De las implementaciones existentes se pueden extraer las siguientes conclusiones:

\begin{itemize}
    \item La ubicación estratégica es crítica: zonas de alta concentración de tráfico peatonal
    \item La visualización de la energía generada aumenta el engagement de usuarios
    \item El mantenimiento preventivo es esencial para garantizar durabilidad
    \item La integración con sistemas de bajo consumo (LED) maximiza el impacto
    \item El valor simbólico y educativo puede ser tan importante como la generación energética
\end{itemize}
