% Limitaciones del proyecto

El proyecto enfrenta las siguientes limitaciones:

\subsection*{Limitaciones Técnicas}

\begin{itemize}
    \item \textbf{Eficiencia de conversión:} Los materiales piezoeléctricos actuales tienen una eficiencia de conversión limitada (típicamente 2-5\%), lo que limita la cantidad de energía que puede ser generada.
    
    \item \textbf{Durabilidad de materiales:} La vida útil de los materiales piezoeléctricos bajo condiciones de alto tráfico puede verse afectada por el desgaste continuo.
    
    \item \textbf{Variabilidad del flujo:} La generación de energía depende directamente del flujo de pasajeros, que varía según hora, día y temporada.
\end{itemize}

\subsection*{Limitaciones Económicas}

\begin{itemize}
    \item \textbf{Inversión inicial elevada:} Los costos iniciales de adquisición e instalación pueden ser significativos.
    
    \item \textbf{Período de recuperación:} El retorno de la inversión puede tomar varios años, dependiendo del flujo de pasajeros y los costos energéticos.
\end{itemize}

\subsection*{Limitaciones Operativas}

\begin{itemize}
    \item \textbf{Interferencia con operaciones:} La instalación requiere coordinación cuidadosa para minimizar interrupciones en el servicio.
    
    \item \textbf{Mantenimiento especializado:} Se requiere personal capacitado para el mantenimiento de los sistemas piezoeléctricos.
\end{itemize}

\subsection*{Limitaciones Regulatorias}

\begin{itemize}
    \item Necesidad de aprobaciones y permisos de autoridades competentes
    \item Cumplimiento de normativas de seguridad y construcción
    \item Posibles restricciones en modificaciones de infraestructura patrimonial
\end{itemize}

\subsection*{Limitaciones de Información}

\begin{itemize}
    \item Disponibilidad limitada de datos históricos sobre implementaciones similares en contextos locales
    \item Variabilidad en las estimaciones de rendimiento energético reportadas en la literatura
\end{itemize}
