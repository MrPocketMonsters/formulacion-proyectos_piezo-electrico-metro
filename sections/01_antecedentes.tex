% Antecedentes del proyecto

El crecimiento acelerado de las ciudades ha generado una mayor demanda de sistemas de transporte público eficientes y sostenibles. Los sistemas de metro representan una de las soluciones más efectivas para la movilidad urbana, transportando millones de pasajeros diariamente en las principales ciudades del mundo.

Sin embargo, estos sistemas consumen grandes cantidades de energía eléctrica para su operación, incluyendo tracción de trenes, iluminación, ventilación, escaleras mecánicas, y sistemas de señalización. En el contexto actual de crisis energética y cambio climático, es imperativo buscar alternativas innovadoras que permitan reducir el consumo energético y aprovechar fuentes de energía renovable.

La tecnología piezoeléctrica ha demostrado ser una solución prometedora para la captación de energía en entornos urbanos. Este efecto, descubierto por los hermanos Curie en 1880, permite convertir energía mecánica en energía eléctrica mediante materiales especiales que generan voltaje cuando son sometidos a presión o deformación.

Experiencias internacionales exitosas, como las implementaciones en estaciones de tren en Tokio (Japón), Rotterdam (Países Bajos), y Londres (Reino Unido), han demostrado la viabilidad técnica y económica de esta tecnología en infraestructuras de transporte público de alta afluencia.
