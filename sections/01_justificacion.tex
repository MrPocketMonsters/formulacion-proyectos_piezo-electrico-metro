% Justificación del proyecto

La implementación de tecnología piezoeléctrica en las estaciones del metro se justifica por las siguientes razones:

\subsection*{Justificación Técnica}

\begin{itemize}
    \item Los materiales piezoeléctricos han alcanzado un nivel de madurez tecnológica que permite su aplicación comercial en infraestructuras urbanas.
    \item El alto flujo de pasajeros en las estaciones del metro proporciona una fuente constante y predecible de energía mecánica susceptible de ser capturada.
    \item La tecnología es escalable y puede implementarse de manera gradual en diferentes estaciones.
\end{itemize}

\subsection*{Justificación Económica}

\begin{itemize}
    \item Reducción de costos operativos a largo plazo mediante la generación de energía in-situ.
    \item Los precios de la tecnología piezoeléctrica han disminuido significativamente en la última década, mejorando su viabilidad económica.
    \item Posibilidad de acceder a incentivos gubernamentales y financiamiento para proyectos de energía renovable.
\end{itemize}

\subsection*{Justificación Ambiental}

\begin{itemize}
    \item Contribución a la reducción de emisiones de gases de efecto invernadero.
    \item Promoción de prácticas sostenibles en el transporte público.
    \item Alineación con objetivos de desarrollo sostenible y compromisos climáticos internacionales.
\end{itemize}

\subsection*{Justificación Social}

\begin{itemize}
    \item Generación de conciencia ambiental entre los usuarios del sistema.
    \item Mejora de la imagen institucional del sistema de metro.
    \item Demostración de liderazgo en innovación y sostenibilidad urbana.
\end{itemize}
