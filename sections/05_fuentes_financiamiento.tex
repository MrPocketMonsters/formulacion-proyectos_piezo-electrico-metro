% Fuentes de financiamiento

\subsection{Estructura de Financiamiento Propuesta}

\begin{table}[H]
\centering
\begin{tabular}{@{}lrr@{}}
\toprule
\textbf{Fuente} & \textbf{Monto (USD)} & \textbf{Porcentaje} \\ \midrule
Recursos propios (Metro) & \$75,000 & 30\% \\
Subsidio gubernamental & \$100,000 & 40\% \\
Financiamiento verde & \$76,540 & 30\% \\ \midrule
\textbf{Total} & \textbf{\$251,540} & \textbf{100\%} \\ \bottomrule
\end{tabular}
\caption{Estructura de financiamiento propuesta}
\end{table}

\subsection{Fuentes de Financiamiento Disponibles}

\subsubsection{Recursos Propios}

\begin{itemize}
    \item Presupuesto anual de inversión en infraestructura
    \item Fondos de modernización tecnológica
\end{itemize}

\subsubsection{Subsidios y Apoyo Gubernamental}

\begin{itemize}
    \item Programas de fomento a energías renovables
    \item Fondos de innovación en transporte público
    \item Incentivos fiscales para proyectos sostenibles
    \item Posible apoyo de ministerios de energía y medio ambiente
\end{itemize}

\subsubsection{Financiamiento Verde (Green Bonds)}

\begin{itemize}
    \item Bonos verdes institucionales
    \item Líneas de crédito con tasa preferencial para proyectos sostenibles
    \item Condiciones típicas:
        \begin{itemize}
            \item Tasa: 4-6\% anual
            \item Plazo: 10-15 años
            \item Período de gracia: 2 años
        \end{itemize}
\end{itemize}

\subsubsection{Financiamiento Internacional}

\begin{itemize}
    \item Banco Mundial - Programa de Energía Sostenible
    \item Banco Interamericano de Desarrollo (BID)
    \item Fondo Verde para el Clima (GCF)
    \item Cooperación internacional bilateral
\end{itemize}

\subsubsection{Alianzas Público-Privadas (PPP)}

\begin{itemize}
    \item Participación de fabricantes de tecnología piezoeléctrica
    \item Esquemas de financiamiento compartido
    \item Modelos de ESCo (Energy Service Company)
\end{itemize}

\subsection{Incentivos Aplicables}

\begin{itemize}
    \item Exención de impuestos para equipos de generación renovable
    \item Depreciación acelerada de activos
    \item Créditos fiscales por reducción de emisiones
    \item Certificados de energía renovable (RECs)
\end{itemize}

\subsection{Estrategia de Financiamiento Recomendada}

\begin{enumerate}
    \item Solicitar subsidio gubernamental (40\% del CAPEX)
    \item Comprometer recursos propios para demostrar compromiso institucional (30\%)
    \item Complementar con financiamiento verde preferencial (30\%)
    \item Explorar beneficios fiscales adicionales para mejorar flujo de caja
\end{enumerate}
