% Tecnologías de captación de energía

\subsection{Energy Harvesting Cinético}

La captación de energía cinética convierte el movimiento mecánico en electricidad utilizando diferentes principios físicos.

\subsubsection{Tecnología Piezoeléctrica}

Utiliza materiales piezoeléctricos que generan voltaje al ser comprimidos o flexionados.

\textbf{Ventajas:}
\begin{itemize}
    \item Alta densidad de potencia
    \item Sin partes móviles
    \item Bajo mantenimiento
    \item Respuesta rápida
\end{itemize}

\textbf{Desventajas:}
\begin{itemize}
    \item Voltaje alto, corriente baja
    \item Requiere circuitos de acondicionamiento
    \item Limitada eficiencia de conversión
\end{itemize}

\subsubsection{Tecnología Electromagnética}

Utiliza el movimiento relativo entre bobinas y magnetos para generar electricidad.

\textbf{Ventajas:}
\begin{itemize}
    \item Mayor corriente generada
    \item Tecnología madura
    \item Voltaje más bajo y estable
\end{itemize}

\textbf{Desventajas:}
\begin{itemize}
    \item Partes móviles sujetas a desgaste
    \item Mayor volumen
    \item Mantenimiento más frecuente
\end{itemize}

\subsubsection{Tecnología Electrostática}

Utiliza condensadores variables para convertir movimiento en electricidad.

\subsection{Comparación de Tecnologías}

\begin{table}[H]
\centering
\begin{tabular}{@{}lccc@{}}
\toprule
\textbf{Característica} & \textbf{Piezoeléctrica} & \textbf{Electromagnética} & \textbf{Electrostática} \\ \midrule
Densidad de potencia & Alta & Media & Baja \\
Durabilidad & Muy alta & Media & Alta \\
Complejidad & Media & Baja & Alta \\
Costo & Medio & Bajo & Alto \\
Escalabilidad & Excelente & Buena & Limitada \\ \bottomrule
\end{tabular}
\caption{Comparación de tecnologías de energy harvesting}
\end{table}
