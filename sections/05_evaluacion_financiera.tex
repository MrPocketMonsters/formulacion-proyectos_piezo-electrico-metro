% Evaluación financiera

\subsection{Indicadores Financieros}

\subsubsection{Valor Actual Neto (VAN)}

\begin{equation}
VAN = \sum_{t=0}^{n} \frac{FC_t}{(1+r)^t}
\end{equation}

donde:
\begin{itemize}
    \item $FC_t$ = Flujo de caja en el período $t$
    \item $r$ = Tasa de descuento (8\%)
    \item $n$ = Horizonte de evaluación (20 años)
\end{itemize}

\textbf{Resultado: VAN = \$-15,320}

\subsubsection{Tasa Interna de Retorno (TIR)}

La TIR es la tasa de descuento que hace el VAN igual a cero.

\textbf{Resultado: TIR = 7.2\%}

\subsubsection{Período de Recuperación (Payback)}

Tiempo requerido para recuperar la inversión inicial.

\textbf{Resultado: Payback = 18 años (aproximadamente)}

\subsubsection{Relación Beneficio-Costo (B/C)}

\begin{equation}
B/C = \frac{\sum Beneficios\ Descontados}{\sum Costos\ Descontados}
\end{equation}

\textbf{Resultado: B/C = 0.94}

\subsection{Interpretación de Resultados}

\begin{table}[H]
\centering
\begin{tabular}{@{}lcc@{}}
\toprule
\textbf{Indicador} & \textbf{Valor} & \textbf{Interpretación} \\ \midrule
VAN & -\$15,320 & Ligeramente negativo \\
TIR & 7.2\% & Bajo la tasa de descuento \\
Payback & 18 años & Período largo \\
B/C & 0.94 & Cercano al punto de equilibrio \\ \bottomrule
\end{tabular}
\caption{Resumen de indicadores financieros}
\end{table}

\subsection{Análisis de Viabilidad Económica}

Los indicadores financieros muestran que el proyecto, considerado únicamente desde la perspectiva económica tradicional, presenta:

\begin{itemize}
    \item Viabilidad financiera marginal con los supuestos conservadores utilizados
    \item Sensibilidad alta a variaciones en flujo de pasajeros y tarifa eléctrica
    \item Justificación más robusta cuando se consideran beneficios intangibles (imagen, sostenibilidad, innovación)
\end{itemize}

\textbf{Recomendación:} El proyecto se justifica principalmente por sus beneficios ambientales, sociales y estratégicos, más que por el retorno financiero directo. Se sugiere explorar fuentes de financiamiento concesional o incentivos gubernamentales para mejorar la viabilidad económica.
