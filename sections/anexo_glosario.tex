% Glosario de términos

\begin{description}
    \item[Piezoelectricidad] Propiedad de ciertos materiales de generar carga eléctrica cuando son sometidos a esfuerzos mecánicos.
    
    \item[Energy Harvesting] Captación de energía de fuentes ambientales (vibraciones, luz, calor) para convertirla en energía eléctrica utilizable.
    
    \item[Coeficiente piezoeléctrico ($d_{ij}$)] Parámetro que cuantifica la capacidad de un material para convertir energía mecánica en eléctrica.
    
    \item[Factor de acoplamiento ($k$)] Medida de la eficiencia de conversión de energía entre los dominios mecánico y eléctrico.
    
    \item[PZT] Siglas de Titanato Zirconato de Plomo, la cerámica piezoeléctrica más utilizada comercialmente.
    
    \item[PVDF] Siglas de Fluoruro de Polivinilideno, un polímero piezoeléctrico flexible.
    
    \item[Loseta piezoeléctrica] Dispositivo modular que integra materiales piezoeléctricos y puede instalarse en pisos para captar energía del tránsito peatonal.
    
    \item[MPPT] Maximum Power Point Tracking, algoritmo para optimizar la extracción de energía de sistemas generadores.
    
    \item[Supercapacitor] Dispositivo de almacenamiento de energía con alta densidad de potencia, adecuado para cargas y descargas rápidas.
    
    \item[VAN] Valor Actual Neto, indicador financiero que mide la rentabilidad de un proyecto en valor presente.
    
    \item[TIR] Tasa Interna de Retorno, tasa de descuento que hace que el VAN de un proyecto sea igual a cero.
    
    \item[Huella de carbono] Cantidad total de emisiones de gases de efecto invernadero producidas directa o indirectamente por una actividad.
    
    \item[kWh] Kilowatt-hora, unidad de energía equivalente al consumo de un kilowatt de potencia durante una hora.
    
    \item[Efecto inverso] Fenómeno por el cual un material piezoeléctrico se deforma al aplicarle un campo eléctrico.
    
    \item[Flujo peatonal] Cantidad de personas que transitan por una zona específica en un período de tiempo determinado.
    
    \item[Smart Grid] Red eléctrica inteligente que utiliza tecnología digital para optimizar la generación, distribución y consumo de energía.
    
    \item[IoT] Internet of Things (Internet de las Cosas), red de dispositivos físicos conectados que intercambian datos.
    
    \item[LED] Light Emitting Diode (Diodo Emisor de Luz), tecnología de iluminación de alta eficiencia energética.
\end{description}
