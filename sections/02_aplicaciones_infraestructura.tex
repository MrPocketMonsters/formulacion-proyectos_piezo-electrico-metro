% Aplicaciones de piezoelectricidad en infraestructura

\subsection{Aplicaciones en Transporte}

\subsubsection{Pavimentos Inteligentes}

Instalación de sistemas piezoeléctricos en:
\begin{itemize}
    \item Autopistas y carreteras
    \item Estacionamientos
    \item Puentes vehiculares
\end{itemize}

\subsubsection{Infraestructura Ferroviaria}

\begin{itemize}
    \item Generación de energía mediante vibración de vías férreas
    \item Monitoreo estructural de rieles
    \item Sistemas de señalización autoalimentados
\end{itemize}

\subsubsection{Aeropuertos}

\begin{itemize}
    \item Pisos de terminales de pasajeros
    \item Pistas de aterrizaje (monitoreo)
    \item Sistemas de iluminación de emergencia
\end{itemize}

\subsection{Aplicaciones Urbanas}

\subsubsection{Espacios Peatonales}

\begin{itemize}
    \item Plazas públicas
    \item Aceras de alto tráfico
    \item Cruces peatonales
    \item Parques y áreas recreativas
\end{itemize}

\subsubsection{Edificios Inteligentes}

\begin{itemize}
    \item Pisos de oficinas
    \item Centros comerciales
    \item Estadios y eventos masivos
    \item Gimnasios y centros deportivos
\end{itemize}

\subsection{Otras Aplicaciones}

\begin{itemize}
    \item Sensores estructurales autoalimentados
    \item Sistemas de monitoreo remoto
    \item Dispositivos IoT sin baterías
    \item Señalización vial inteligente
\end{itemize}
